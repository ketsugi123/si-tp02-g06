%
% Anexo 2
%
\chapter{Código das Classes do \textit{Package} \texttt{model.tables}} \label{a3}
\begin{minted}
[
    frame=lines,
    framesep=2mm,
    baselinestretch=1.2,
    fontsize=\footnotesize,
    linenos
]
{Java}
package model.tables;

import jakarta.persistence.*;
import model.relations.Compra;

import java.io.Serial;
import java.io.Serializable;
import java.util.ArrayList;
import java.util.List;

@Entity
@Table(name="jogo")
public class Jogo implements Serializable {

    @Serial
    private static final long serialVersionUID = 1L;

    @Id
    @GeneratedValue(strategy = GenerationType.UUID)
    private String id;
    private String nome;
    private String url;

    @OneToMany(mappedBy = "jogo")
    private List<Compra> compras = new ArrayList<>();

    public void addCompra(Compra compra){ this.compras.add(compra); }

    public List<Compra> getCompras() { return compras; }


    // 1:N pertence
    @OneToMany(mappedBy="jogo",cascade=CascadeType.PERSIST, orphanRemoval=true)
    private List<Cracha> crachas = new ArrayList<>();

    public List<Cracha> getCrachas() {
        return crachas;
    }

    public void addCracha(Cracha cracha) {
        this.crachas.add(cracha);
    }
    @OneToMany(mappedBy = "jogo", cascade = CascadeType.PERSIST, orphanRemoval=true)
    private List<Partida> partidas = new ArrayList<>();

    public List<Partida> getPartidas() {
        return partidas;
    }

    public void addPartida(Partida partida){
        partidas.add(partida);
    }

    public String getId(){ return this.id; }
    public String getNome(){ return this.nome; }
    public void setNome(String nome) { this.nome = nome; }

    public String getUrl(){ return this.url; }
    public void setUrl(String url) { this.url = url; }

}
\end{minted}
\captionof{listing}{Código da classe Jogo\label{lst:Jogo}}

\begin{minted}
[
    frame=lines,
    framesep=2mm,
    baselinestretch=1.2,
    fontsize=\footnotesize,
    linenos
]
{Java}
package model.tables;

import jakarta.persistence.*;
import model.relations.Compra;
import model.relations.CrachasAdquiridos;
import model.relations.Participa;

import java.io.Serial;
import java.io.Serializable;
import java.util.ArrayList;
import java.util.List;

@Entity
@Table(name="jogador")
public class Jogador implements Serializable {
    @Serial
    private static final long serialVersionUID = 1L;

    public Jogador(){ }


    @Id
    @GeneratedValue(strategy = GenerationType.IDENTITY)
    private int id;
    private String email;
    private String username;
    private String estado;
    private String regiao;
    @OneToMany(mappedBy = "jogador")
    private List<Compra> compras = new ArrayList<>();
    public void addCompra(Compra compra){ this.compras.add(compra); }
    public List<Compra> getCompras() { return compras; }


    //N:N Crachas

    @OneToMany(mappedBy = "jogador")
    private List<CrachasAdquiridos> crachasAdquiridosList = new ArrayList<>();

    public List<CrachasAdquiridos> getCrachasList() {
        return crachasAdquiridosList;
    }

    public void adquirirCracha(CrachasAdquiridos crachasAdquiridos) {
        this.crachasAdquiridosList.add(crachasAdquiridos);
    }


    //N:N Participa Relation
    @OneToMany(mappedBy = "jogador")
    private List<Participa> participaList = new ArrayList<>();
    public List<Participa> getParticipaList() { return this.participaList; }
    public void addParticipacao(Participa participa){this.participaList.add(participa);}

    @OneToMany(mappedBy = "jogador", cascade = CascadeType.PERSIST)
    private List<Partida_Normal> partidasNormais = new ArrayList<>();

    public List<Partida_Normal> getPartidasNormais() {
        return partidasNormais;
    }

    public void addPartida_Normal(Partida_Normal partida_normal){
        this.partidasNormais.add(partida_normal);
    }

    @OneToMany(mappedBy = "jogador", cascade = CascadeType.PERSIST)
    private List<Partida_MultiJogador> partidasMultiJogador = new ArrayList<>();

    public void addPartidaMultiJogador(Partida_MultiJogador partida){
        this.partidasMultiJogador.add(partida);
    }
    public List<Partida_MultiJogador> getPartidasMultiJogador() {
        return partidasMultiJogador;
    }



    //getter and setters
    public int getId(){
        return this.id;
    }

    public String getEmail(){
        return this.email;
    }

    public void setEmail(String email){
        this.email = email;
    }

    public String getUsername(){
        return this.username;
    }

    public void setUsername(String username){
        this.username = username;
    }

    public String getEstado(){
        return this.estado;
    }

    public void setEstado(String estado){
        this.estado = estado;
    }

    public String getRegiao(){
        return this.regiao;
    }

    public void setRegiao(String regiao){
        this.regiao = regiao;
    }

}
\end{minted}
\captionof{listing}{Código da classe Jogador\label{lst:Jogador}}

\begin{minted}
[
    frame=lines,
    framesep=2mm,
    baselinestretch=1.2,
    fontsize=\footnotesize,
    linenos
]
{Java}
package model.tables;

import jakarta.persistence.*;
import model.embeddables.PartidaId;

import java.io.Serial;
import java.io.Serializable;
import java.util.Date;

@Entity
@Table(name="partida")
public class Partida implements Serializable {

    @Serial
    private static final long serialVersionUID = 1L;

    public Partida() { }

    @EmbeddedId
    private PartidaId id;

    String estado;
    String nome_regiao;

    Date dtinicio;

    Date dtfim;

    // 1:N
    @ManyToOne
    @MapsId
    @JoinColumn(name = "id")
    private Jogo jogo;

    public void setJogo(Jogo jogo) {
        this.jogo = jogo;
    }

    public Jogo getJogo() {
        return jogo;
    }

    public PartidaId getId() {
        return id;
    }

    @OneToOne(mappedBy = "partida")
    private Partida_MultiJogador partida_multiJogador;

    public Partida_MultiJogador getPartida_multiJogador() {
        return partida_multiJogador;
    }

    public void setPartida_multiJogador(Partida_MultiJogador partida_multiJogador) {
        this.partida_multiJogador = partida_multiJogador;
    }

    @OneToOne(mappedBy = "partida")
    private Partida_Normal partida_normal;

    public Partida_Normal getPartida_normal() {
        return partida_normal;
    }

    public void setPartida_normal(Partida_Normal partida_normal) {
        this.partida_normal = partida_normal;
    }

    public void setId(PartidaId id) {
        this.id = id;
    }

    public String getEstado() {
        return estado;
    }

    public void setEstado(String estado) {
        this.estado = estado;
    }

    public String getNome_regiao() {
        return nome_regiao;
    }

    public void setNome_regiao(String nome_regiao) {
        this.nome_regiao = nome_regiao;
    }

    public Date getDtinicio() {
        return dtinicio;
    }

    public void setDtinicio(Date dtinicio) {
        this.dtinicio = dtinicio;
    }

    public void setDtfim(Date dtfim) {
        this.dtfim = dtfim;
    }

    public Date getDtfim() {
        return dtfim;
    }
}
\end{minted}
\captionof{listing}{Código da classe Partida\label{lst:Partida}}

\begin{minted}
[
    frame=lines,
    framesep=2mm,
    baselinestretch=1.2,
    fontsize=\footnotesize,
    linenos
]
{Java}
package model.tables;

import jakarta.persistence.*;
import model.embeddables.PartidaMId;
import model.embeddables.PartidaNId;

import java.io.Serial;
import java.io.Serializable;

@Entity
@Table(name="partida_multijogador")
public class Partida_MultiJogador implements Serializable {
    @Serial
    private static final long serialVersionUID = 1L;

    public Partida_MultiJogador() { }

    @EmbeddedId
    private PartidaMId partida_multiId;

    int pontuacao;

    @OneToOne
    @JoinColumns({
            @JoinColumn(name = "partida", referencedColumnName = "partida", insertable = false, updatable = false),
            @JoinColumn(name = "jogo", referencedColumnName = "jogo", insertable = false, updatable = false)
    })
    private Partida partida;


    @ManyToOne
    @MapsId
    @JoinColumn(name = "jogador", referencedColumnName = "jogador")
    private Jogador jogador;

    public Jogador getJogador() {
        return jogador;
    }

    public void setJogador(Jogador jogador) {
        this.jogador = jogador;
    }

    public Partida getPartida() {
        return partida;
    }

    public void setPontuacao(int pontuacao) {
        this.pontuacao = pontuacao;
    }

    public void setPartida_multiId(PartidaMId partida) {
        this.partida_multiId = partida;
    }

    public PartidaMId getPartida_multiId() {
        return partida_multiId;
    }

    public int getPontuacao() {
        return pontuacao;
    }

}
\end{minted}
\captionof{listing}{Código da classe Partida\_MultiJogador\label{lst:Partida_MultiJogador}}

\begin{minted}
[
    frame=lines,
    framesep=2mm,
    baselinestretch=1.2,
    fontsize=\footnotesize,
    linenos
]
{Java}
package model.tables;

import jakarta.persistence.*;
import model.embeddables.PartidaNId;

import java.io.Serial;
import java.io.Serializable;

@Entity
@Table(name="partida_normal")
public class Partida_Normal extends Partida implements Serializable {

    @Serial
    private static final long serialVersionUID = 1L;

    public Partida_Normal() { }

    @EmbeddedId
    private PartidaNId partida_normalId;

    private int pontuacao;

    @OneToOne
    @JoinColumns({
            @JoinColumn(name = "partida", referencedColumnName = "partida", insertable = false, updatable = false),
            @JoinColumn(name = "jogo", referencedColumnName = "jogo", insertable = false, updatable = false)
    })
    private Partida partida;

    public Partida getPartida() {
        return partida;
    }

    @ManyToOne
    @MapsId
    @JoinColumn(name = "jogador", referencedColumnName = "jogador")
    private Jogador jogador;

    public Jogador getJogador() {
        return jogador;
    }

    public void setJogador(Jogador jogador) {
        this.jogador = jogador;
    }

    public PartidaNId getPartida_normalId() {
        return partida_normalId;
    }

    public void setPartida_normalId(PartidaNId partida) {
        this.partida_normalId = partida;
    }

    public int getPontuacao() {
        return pontuacao;
    }

    public void setPontuacao(int pontuacao) {
        this.pontuacao = pontuacao;
    }


}
\end{minted}
\captionof{listing}{Código da classe Partida\_Normal\label{lst:Partida_Normal}}

\begin{minted}
[
    frame=lines,
    framesep=2mm,
    baselinestretch=1.2,
    fontsize=\footnotesize,
    linenos
]
{Java}
package model.tables;


import jakarta.persistence.*;
import model.embeddables.CrachaId;
import model.relations.CrachasAdquiridos;

import java.io.Serial;
import java.io.Serializable;
import java.util.ArrayList;
import java.util.List;

@Table(name="cracha")
@Entity
public class Cracha implements Serializable {

    @Serial
    private static final long serialVersionUID = 1L;

    public static long getSerialVersionUID() {
        return serialVersionUID;
    }

    @EmbeddedId
    private CrachaId id;

    int limite;

    String url;

    @ManyToOne
    @MapsId
    @JoinColumn(name="jogo")
    private Jogo jogo;

    public Jogo getJogo() {
        return jogo;
    }

    public void setJogo(Jogo jogo) {
        this.jogo = jogo;
    }

    @OneToMany(mappedBy = "cracha")
    private List<CrachasAdquiridos> crachasAdquiridosList = new ArrayList<>();

    public List<CrachasAdquiridos> getCrachasList() {
        return crachasAdquiridosList;
    }

    public void addCrachas(CrachasAdquiridos crachasAdquiridos) {
        this.crachasAdquiridosList.add(crachasAdquiridos);
    }


    public CrachaId getId() {
        return this.id;
    }

    public void setId(CrachaId id) {
        this.id = id;
    }

    public int getLimite() {
        return limite;
    }

    public void setLimite(int limit) {
        this.limite = limit;
    }

    public String getUrl() {
        return url;
    }

    public void setUrl(String url) {
        this.url = url;
    }

}
\end{minted}
\captionof{listing}{Código da classe Cracha\label{lst:Cracha}}

\begin{minted}
[
    frame=lines,
    framesep=2mm,
    baselinestretch=1.2,
    fontsize=\footnotesize,
    linenos
]
{Java}
package model.tables;

import jakarta.persistence.*;
import model.relations.Participa;

import java.io.Serial;
import java.io.Serializable;
import java.util.ArrayList;
import java.util.List;

@Entity
@Table(name="conversa")
public class Conversa implements Serializable {

    @Serial
    private static final long serialVersionUID = 1L;

    public Conversa() { }

    @Id
    @GeneratedValue(strategy = GenerationType.IDENTITY)
    private int id;
    private String nome;

    // N:N relation participa
    @OneToMany(mappedBy = "conversa")
    private List<Participa> participantes = new ArrayList<>();

    public List<Participa> getParticipantes(){ return this.participantes; }

    public void addJogador(Participa participante){
        this.participantes.add(participante);
    }

    // 1:N relation contem
    @OneToMany(mappedBy = "conversa", cascade = CascadeType.PERSIST, orphanRemoval = true)
    private List<Mensagem> mensagens = new ArrayList<>();

    public List<Mensagem> getMensagens() {
        return mensagens;
    }

    public void addMensagem(Mensagem mensagem) {
        this.mensagens.add(mensagem);
    }

    //getter and setters
    public int getId() { return id; }
    public void setNome(String nome){ this.nome = nome; }
    public String getNome(){ return this.nome; }
}
\end{minted}
\captionof{listing}{Código da classe Conversa\label{lst:Conversa}}

\begin{minted}
[
    frame=lines,
    framesep=2mm,
    baselinestretch=1.2,
    fontsize=\footnotesize,
    linenos
]
{Java}
package model.tables;

import jakarta.persistence.*;
import model.embeddables.MensagemId;

import java.io.Serial;
import java.io.Serializable;
import java.sql.Date;

@Entity
@Table(name = "mensagem")
public class Mensagem implements Serializable {

    @Serial
    private static final long serialVersionUID = 1L;
    public Mensagem() { }

    @EmbeddedId
    private MensagemId mensagemId;

    String conteudo;

    Date data;


    @ManyToOne
    @MapsId
    @JoinColumn(name="conversa")
    private Conversa conversa;

    public MensagemId getMensagemId() {
        return mensagemId;
    }

    public void setMensagemId(MensagemId mensagemId) {
        this.mensagemId = mensagemId;
    }

    public Date getData() {
        return data;
    }

    public void setData(Date data) {
        this.data = data;
    }

    public String getConteudo() {
        return conteudo;
    }

    public void setConteudo(String conteudo) {
        this.conteudo = conteudo;
    }
}
\end{minted}
\captionof{listing}{Código da classe Mensagem\label{lst:Mensagem}}