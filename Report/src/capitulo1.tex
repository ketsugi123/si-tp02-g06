%
% Capítulo 1
%
\chapter{Introdução} \label{cap1}

Uma base de dados por ela mesma tem o seu mérito, quando bem estruturada e implementada é uma mais valia para qualquer empresa que procure organizar a sua informação. No entanto
esta não deve permanecer sozinha, se for acompanhada por uma aplicação capaz de comunicar numa outra linguagem de programação, a base de dados pode ser utilizada por aplicações
e programas de computador com mais flexibilidade no seu uso.

Por forma a cumprir este objetivo foi desenvolvida uma aplicação \textit{Java} que faz uso da biblioteca \texttt{JPA}, através da qual irá criar um modelo de dados e comunicar
com a base de dados remota.

%
% Secção 1.1
%
\section{Objetivos} \label{sec11}

Em primeiro lugar é necessário implementar o modelo de dados ao nível aplicacional. Para este efeito a biblioteca \texttt{JPA} permite usar anotações do \textit{Java} para determinar
o tipo de casa campo de uma classe e como este deve interagir com os tipos de \textit{PostgreSQL} \cite{postgresql:docs}.

Esta aplicação deverá ser capaz de permitir a um programador que a utilize de aceder a todas as funcionalidades presentes na fase 1, logo deverá implementar funções que façam uso
das funções, procedimentos armazenados, vistas e gatilhos presentes no modelo de base de dados.

Para além de apresentar uma interface para acesso à base de dados esta aplicação deve também permitir ao programador escolher certas funcionalidades entre \textit{optimistic locking}
e \textit{pessimistic locking}.
