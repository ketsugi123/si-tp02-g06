%
% Anexo 2
%
\chapter{Código das Classes do \textit{Package} \texttt{model.relations}} \label{a2}

Estas listagens contêm código correspondente às classes presentes no \textit{package} \texttt{relations}

\begin{minted}
[
    frame=lines,
    framesep=2mm,
    baselinestretch=1.2,
    fontsize=\footnotesize,
    linenos
]
{Java}
package model.relations;

import jakarta.persistence.*;
import model.tables.Jogador;
import model.tables.Jogo;
import model.embeddables.CompraId;

import java.io.Serial;
import java.io.Serializable;
import java.util.Date;

@Entity
@Table(name="compra")
public class Compra implements Serializable {

    @Serial
    private static final long serialVersionUID = 1L;

    public Compra() { }

    @EmbeddedId
    private CompraId id;
    private Date data;
    private Double preco;


    // N:N relation Compra between jogo and jogador
    @ManyToOne
    @MapsId("jogo")
    @JoinColumn(name = "jogo")
    private Jogo jogo;

    @ManyToOne
    @MapsId("jogador")
    @JoinColumn(name = "jogador")
    private Jogador jogador;

    //getter and setters
    public Jogador getJogador() { return jogador; }

    public void setJogador(Jogador jogador) { this.jogador = jogador; }

    public Jogo getJogo() { return jogo; }

    public void setJogo(Jogo jogo) { this.jogo = jogo; }

    public CompraId getId() { return this.id; }

    public Date getData() { return data; }

    public void setData(Date data) { this.data = data; }

    public void setPreco(Double preco) { this.preco = preco;}
    public Double getPreco() { return preco; }

}
\end{minted}
\captionof{listing}{Código da classe Compra\label{lst:Compra}}

\begin{minted}
[
    frame=lines,
    framesep=2mm,
    baselinestretch=1.2,
    fontsize=\footnotesize,
    linenos
]
{Java}
package model.relations;

import jakarta.persistence.*;
import model.tables.Cracha;
import model.tables.Jogador;
import model.embeddables.CrachasAdquiridosId;

import java.io.Serializable;

@Entity
@Table(name="crachá_jogador")
public class CrachasAdquiridos implements Serializable {

    @EmbeddedId
    private CrachasAdquiridosId id;


    // N:N realtion CrachasAdquiridos
    @ManyToOne
    @MapsId("jogador")
    @JoinColumn(name="jogador",  referencedColumnName = "jogador")
    private Jogador jogador;


    @ManyToOne
    @JoinColumns({
            @JoinColumn(name = "cracha_nome", referencedColumnName = "nome"),
            @JoinColumn(name = "cracha_jogador", referencedColumnName = "jogador")
    })
    private Cracha cracha;


    public CrachasAdquiridosId getId() {
        return id;
    }

    public void setId(CrachasAdquiridosId id) {
        this.id = id;
    }

    public Cracha getCracha() {
        return cracha;
    }

    public void setCracha(Cracha cracha) {
        this.cracha = cracha;
    }

    public Jogador getJogador() {
        return jogador;
    }

    public void setJogador(Jogador jogador) {
        this.jogador = jogador;
    }
}
\end{minted}
\captionof{listing}{Código da classe CrachasAdquiridos\label{lst:CrachasAdquiridos}}

\begin{minted}
[
    frame=lines,
    framesep=2mm,
    baselinestretch=1.2,
    fontsize=\footnotesize,
    linenos
]
{Java}
package model.relations;

import jakarta.persistence.*;
import model.tables.Conversa;
import model.tables.Jogador;
import model.embeddables.ParticipaId;

import java.io.Serial;
import java.io.Serializable;

@Entity
@Table(name = "jogador_conversa")
public class Participa implements Serializable {

    @Serial
    private static final long serialVersionUID = 1L;

    public Participa(){ }
    @EmbeddedId
    private ParticipaId id;

    @ManyToOne
    @MapsId("jogador")
    @JoinColumn(name="jogador")
    private Jogador jogador;

    @ManyToOne
    @MapsId("conversa")
    @JoinColumn(name="conversa")
    private Conversa conversa;


    //getters and setters
    public ParticipaId getId() {return id;}

    public void setId(ParticipaId id) { this.id = id; }

    public Jogador getJogador() { return jogador; }

    public void setJogador(Jogador jogador) { this.jogador = jogador; }

    public Conversa getConversa() { return conversa; }

    public void setConversa(Conversa conversa) { this.conversa = conversa; }
}
\end{minted}
\captionof{listing}{Código da classe Participa\label{lst:Participa}}