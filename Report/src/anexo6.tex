%
% Anexo 2
%
\chapter{Código das Classe \texttt{App}} \label{a6}
\begin{minted}
[
    frame=lines,
    framesep=2mm,
    baselinestretch=1.2,
    fontsize=\footnotesize,
    linenos
]
{Java}
/*
 Walter Vieira (2022-04-22)
 Sistemas de Informação - projeto JPAAulas_ex3
 Código desenvolvido para iulustração dos conceitos sobre acesso a dados, concretizados com base na especificação JPA.
 Todos os exemplos foram desenvolvidos com EclipseLinlk (3.1.0-M1), usando o ambientre Eclipse IDE versão 2022-03 (4.23.0).
 
Não existe a pretensão de que o código estaja completo.

Embora tenha sido colocado um esforço significativo na correção do código, não há garantias de que ele não contenha erros que possam 
acarretar problemas vários, em particular, no que respeita à consistência dos dados.  
 
*/

package presentation;

import java.util.Scanner;


import businessLogic.*;


/**
 * Hello world!
 *
 */

public class App {
	protected interface ITest {
		void test();
	}

	public static void main( String[] args ) {
		BLService srv = new BLService();
		Scanner imp = new Scanner(System.in);
		ITest[] tests = new ITest[] {
				() -> {
					System.out.println("\n\n\n ------------------- CREATE PLAYER --------------------");
					System.out.println("Introduza um username");
					String playerName = imp.nextLine();
					System.out.println("Introduza um email");
					String email = imp.nextLine();
					try { System.out.println( srv.createPlayer(playerName, email, "EU") ); }
					catch(Exception e) { System.out.println(e); }
				},
				() -> {
					System.out.println("\n\n\n ------------------- BAN PLAYER --------------------");
					System.out.println("Introduza um email de um jogador que queira banir");
					String email = imp.nextLine();
					try { System.out.println(srv.setPlayerState(email, "Banido")); }
					catch(Exception e) { System.out.println(e); }
				},
				() -> {
					System.out.println("\n\n\n ------------------- PONTOS POR JOGADOR --------------------");
					System.out.println("Introduza um email de um jogador que queira saber o número total de pontos");
					String email = imp.nextLine();
					try { System.out.println(srv.totalPontosJogador(email)); }
					catch(Exception e) { System.out.println(e); }
				},
				() -> {
					System.out.println("\n\n\n ------------------- TOTAL JOGOS JOGADOR --------------------");
					System.out.println("Introduza um email de um jogador que queira saber o número total de jogos");
					String email = imp.nextLine();
					try { System.out.println(srv.totalJogosJogador(email)); }
					catch(Exception e) {System.out.println(e);}
				},
				() -> {
					System.out.println("\n\n\n ------------------- PONTOS JOGOS POR JOGADOR --------------------");
					try { System.out.println(srv.PontosJogosPorJogador("Valorant")); }
					catch (Exception e){ System.out.println(e);}
				},
				() -> {
					System.out.println("\n\n\n ------------------- ASSOCIAR CRACHA --------------------");
					try { srv.associarCracha(2, "Genshin Impact", "Master");}
					catch (Exception e){ System.out.println(e); }
				},
				() -> {
					System.out.println("\n\n\n ------------------- INICIAR CONVERSA --------------------");
					try { srv.iniciarConversa(1, "newConvo");}
					catch (Exception e){ System.out.println(e); }
				},
				() -> {
					System.out.println("\n\n\n ------------------- JUNTAR CONVERSA --------------------");
					try { srv.juntarConversa(2, 2);}
					catch (Exception e){ System.out.println(e); }
				},
				() -> {
					System.out.println("\n\n\n ------------------- ENVIAR MENSAGEM --------------------");
					try { srv.enviarMensagem(2, 1, "Hello");}
					catch (Exception e){ System.out.println(e); }
				},
				() -> {
					System.out.println("\n\n\n ------------------- JOGADOR TOTAL INFO --------------------");
					try { srv.jogadorTotalInfo(); }
					catch (Exception e) { System.out.println(e); }
				},
				() -> {
					System.out.println("\n\n\n ------------------- ASSOCIAR CRACHA MODEL --------------------");
					try { srv.associarCrachaModel(1, "Valorant", "Pro player");}
					catch (Exception e){ System.out.println(e); }
				},
		  };

		while(true){
			Scanner opt = new Scanner(System.in);
			System.out.printf("Choose a test (1-%d)? -1 to break ",tests.length);
			int option = opt.nextInt();
			if (option >= 1 && option <= tests.length)
				tests[--option].test();
			if(option == -1) break;
		}
    }
}
\end{minted}
\captionof{listing}{Código da classe App\label{lst:App}}
