%
% Capítulo 3
%
\chapter{Funcionalidade da Aplicação} \label{cap3}

Uma vez o modelo feito é agora possível proceder à implementação das funcionalidades desejadas do trabalho. Estas requerem primeiro acesso às funções e procedimentos
armazenados criados na primeira parte, os quais vêm em formato das alíneas do primeiro enunciado 2d até 2l.

Com o acesso a estas funcionalidades obtido, procede-se à implementação da alínea 2h como na fase 1 mas sem usar qualquer procedimento armazenado ou função pgSql,
ou seja, limitando as possibilidades para usar apenas interações entre \texttt{JPA} e PostgreSQL. Após esta implementação é realizada outra vez mas sem a limitação,
o qual permite facilmente comparar as duas implementações e concluir qual será melhor.

Esta aplicação também permite aumentar em 20\% o número de pontos associados a um crachá, o qual foi implementado com \textit{optimistic locking} e \textit{pessimistic
locking}. A versão que utiliza \textit{optimistic locking} vem acompanhada de testes que verificam que uma mensagem de erro é levantada quando existe uma alteração
concorrente que inviabilize a operação.

%
% Secção 3.1
%
\section{Acesso às Funcionalidades da Base de Dados} \label{sec31}

TODO(Descrever a forma como as funções são chamadas no modelo aplicacional)

%
% Secção 3.2
%
\section{Realização da Funcionalidade 2h} \label{sec32}

TODO(Descrever a funcionalidade)

%
% Secção 3.2.1
%
\subsection{Sem usar procedimentos e funções pgSql} \label{sec321}

TODO(Descrever a implementação sem pgSql)

%
% Secção 3.2.2
%
\subsection{Usando procedimentos e funções pgSql} \label{sec322}

TODO(Descrever a implementação com pgSql)

%
% Secção 3.3
%
\section{Funcionalidade Adicional} \label{sec33}

TODO(Aumentar a 20\% cena)

%
% Secção 3.3.1
%
\subsection{Implementação Com \textit{Optimistic Locking}} \label{sec331}

TODO(Descrever a implementação que usa optimistic locking)

%
% Secção 3.3.2
%
\subsection{Implementação Com \textit{Pessimistic Locking}} \label{sec332}

TODO(Descrever a implementação que usa pessimistic locking)

%
% Secção 3.3.3
%
\subsection{Testes da Funcionalidade}\label{sec333}

TODO(Descrever os testes e como se testou concorrência)