%
% Capítulo 2
%
\chapter{Modelo de dados} \label{cap2}

Antes de passar à implementação das funcionalidades da base de dados é necessário primeiro organizar as tabelas e relações em classes de \textit{Java} correspondentes.
Este modelo está dividido em três \textit{packages} diferentes, todos dentro do \textit{package} \texttt{model}. Dentro deste econtra-se \texttt{embeddables}, o qual
contém classes de valores embebidos, ou seja, identificadores complexos que podem englobar mais que um atríbuto. Encontram-se também o \textit{package} \texttt{relations},
que contém as tabelas que representam relações na base de dados e o \textit{package} \texttt{tables}, o qual contém as tabelas principais.

Cada uma destas classes contém as anotações que sejam necessárias. Realça-se o facto de que as classes dentro do \textit{package} \texttt{relations} têm a anotação
\texttt{@Entity}. Esta característica deve-se ao facto de todas as tabelas da base de dados, incluindo as que representam relações, serem tratadas como entidades.

%
% Secção 2.1
%
\section{Embeddables} \label{sec21}

As classes dentro deste \textit{package} servem como auxílio á implementação das classes mais complexas, pois estas permitem ter identificadores complexos que sejam
representados por mais que um atributo. Todas estas classes incluem o uso da anotação \texttt{@Embeddable}.

A implementação destas classes encontra-se no anexo~\ref{a1}.

\section{Relations} \label{sec22}

Este \textit{package} contém as classes que identificam relações do modelo que foram implementadas como tabelas na base de dados. Estas estão colocadas separadamente
pois não representam uma entidade no modelo ER, mas sim uma relação. Todas contêm as anotações \texttt{@Entity} e \texttt{@Table}, com o valor do campo \texttt{name}
igual à tabela correspondente da base de dados PostgreSQL.

Como todas estas são representadas através de identificadores de outras entidades, todas têm um campo com a anotação \texttt{@EmbeddedId}, sendo este campo do tipo
correspondente do \texttt{package} \texttt{embeddables}. A implementação destas classes encontra-se no anexo~\ref{a2}.

\section{Tables} \label{sec23}

Neste \textit{package} encontram-se as tabelas da base de dados que representam entidades no modelo ER. Estas fazem uso das classes presentes em relations para ser mais
fácil de aceder a certas informações, como, por exemplo, a lista de jogadores que compraram um jogo ou a lista de jogos que um jogador comprou. Estes estão presentes como
campos das classes \texttt{Jogo} e \texttt{Jogador}, respetivamente.

Existem ainda funções que acompanham estas implementações pois todos estes campos estão declarados como \texttt{private}, sendo apenas possível aceder fazendo uso de um
\textit{getter} e alterando o seu valor com um \textit{setter}.

A implementação destas classes encontra-se no anexo~\ref{a3}.