% Classe do documento e parâmetros gerais.
\documentclass[a4paper,openright,oneside,11pt]{report}


%
% Times New Roman font.
%
\usefont{T1}{ptm}{m}{n}
\selectfont


% Packages a utilizar e respetivos par�metros.
\usepackage[utf8]{inputenc}
\usepackage[portuguese]{babel}
\addto{\captionsportuguese}{\renewcommand{\bibname}{Refer\^{e}ncias}}
\addto{\captionsportuguese}{\renewcommand{\contentsname}{\'Indice}}
\addto{\captionsportuguese}{\renewcommand{\appendixname}{Anexo}}
\usepackage{graphicx}
\usepackage{tabularx}
\usepackage{url}
\usepackage[Algoritmo]{algorithm}
\usepackage{algorithmicx}
\usepackage{algpseudocode}
\usepackage{setspace}
\usepackage{caption}
\usepackage{listings}
\usepackage[intoc]{nomencl}
\usepackage{minted}
\usemintedstyle{arduino}
\renewcommand{\algorithmicrequire}{\textbf{Dados: }}
\renewcommand{\algorithmicensure}{\textbf{Resultado: }}
\renewcommand{\arraystretch}{2}

% Definições das dimensões das páginas
\setlength{\textheight}{24.00cm}
\setlength{\textwidth}{15.50cm}
\setlength{\topmargin}{0.35cm}
\setlength{\headheight}{0cm}
\setlength{\headsep}{0cm}
\setlength{\oddsidemargin}{0.25cm}
\setlength{\evensidemargin}{0.25cm}

%
% Times New Roman font.
%
\usefont{T1}{ptm}{m}{n}
\selectfont


%\renewcommand{\baselinestretch}{1}

%
% Página inicial (capa)
%
\title{
   \vspace{-60mm}
   \begin{minipage}[l]{160mm}
      \resizebox{50mm}{!}{\includegraphics{./figures/logo_isel.png}}\\
   \end{minipage}\\
   \vspace{20mm}
		% Título do projeto na forma capitalizada. A primeira letra de cada palavra deve ser maiúscula.
   {\bf Nível Aplicacional da Base de Dados \\ Para a Empresa GameOn}
}

% Nome dos autores (um por linha)
\author{
\begin{tabular}{ll}
             & José Alves  \\
             & Alexandre Severino \\
             & Diogo Carichas \\
\end{tabular}}


\date{
\vspace{80mm}
\begin{tabular}{ll}
  {Orientadores} & Walter Vieira \\
\end{tabular}\\
% Deixar o indicador respetivo em função da versão do relatório.
\vspace{10mm}
Relatório de trabalho prático realizado no âmbito de Sistemas de Informação,\\
do curso de licenciatura em Engenharia Informática e de Computadores\\
Semestre de Verão 2022/2023\\
\vspace{20mm}
Maio de 2023}

\makenomenclature

\begin{document}

\pagenumbering{roman}
\maketitle

\baselineskip 18pt % line spacing: 12pt for single, 18pt for 1 1/2, and 24pt for double spacing

\thispagestyle{empty}
% Fim da contracapa

% Página de resumo em Português
\chapter*{Resumo}

TODO()

{\bf Palavras-chave:} palavras;chave

%% Página de resumo em Inglês
\chapter*{Abstract}

TODO()

{\bf Keywords:} key;words;

% Geração do índice de conteúdos
\tableofcontents

% Geração do índice de figuras
\listoffigures

% Geração do índice de tabelas
\listoftables

\newpage
% Iniciar a numeração de páginas
\setcounter{page}{1}
\pagenumbering{arabic}

% Capitulo 1
%
% Capítulo 1
%
\chapter{Introdução} \label{cap1}

Uma base de dados por ela mesma tem o seu mérito, quando bem estruturada e implementada é uma mais valia para qualquer empresa que procure organizar a sua informação. No entanto
esta não deve permanecer sozinha, se for acompanhada por uma aplicação capaz de comunicar numa outra linguagem de programação, a base de dados pode ser utilizada por aplicações
e programas de computador com mais flexibilidade no seu uso.

Por forma a cumprir este objetivo foi desenvolvida uma aplicação \textit{Java} que faz uso da biblioteca \texttt{JPA}, através da qual irá criar um modelo de dados e comunicar
com a base de dados remota.

%
% Secção 1.1
%
\section{Objetivos} \label{sec11}

Em primeiro lugar é necessário implementar o modelo de dados ao nível aplicacional. Para este efeito a biblioteca \texttt{JPA} permite usar anotações do \textit{Java} para determinar
o tipo de casa campo de uma classe e como este deve interagir com os tipos de \textit{PostgreSQL} \cite{postgresql:docs}.

Esta aplicação deverá ser capaz de permitir a um programador que a utilize de aceder a todas as funcionalidades presentes na fase 1, logo deverá implementar funções que façam uso
das funções, procedimentos armazenados, vistas e gatilhos presentes no modelo de base de dados.

Para além de apresentar uma interface para acesso à base de dados esta aplicação deve também permitir ao programador escolher certas funcionalidades entre \textit{optimistic locking}
e \textit{pessimistic locking}.


% Capitulo 2
%
% Capítulo 2
%
\chapter{Modelo de dados} \label{cap2}

Antes de passar à implementação das funcionalidades da base de dados é necessário primeiro organizar as tabelas e relações em classes de \textit{Java} correspondentes.
Este modelo está dividido em três \textit{packages} diferentes, todos dentro do \textit{package} \texttt{model}. Dentro deste econtra-se \texttt{embeddables}, o qual
contém classes de valores embebidos, ou seja, identificadores complexos que podem englobar mais que um atríbuto. Encontram-se também o \textit{package} \texttt{relations},
que contém as tabelas que representam relações na base de dados e o \textit{package} \texttt{tables}, o qual contém as tabelas principais.

Cada uma destas classes contém as anotações que sejam necessárias. Realça-se o facto de que as classes dentro do \textit{package} \texttt{relations} têm a anotação
\texttt{@Entity}. Esta característica deve-se ao facto de todas as tabelas da base de dados, incluindo as que representam relações, serem tratadas como entidades.

%
% Secção 2.1
%
\section{Embeddables} \label{sec21}

As classes dentro deste \textit{package} servem como auxílio á implementação das classes mais complexas, pois estas permitem ter identificadores complexos que sejam
representados por mais que um atributo. Todas estas classes incluem o uso da anotação \texttt{@Embeddable}.

A implementação destas classes encontra-se no anexo~\ref{a1}.

\section{Relations} \label{sec22}

Este \textit{package} contém as classes que identificam relações do modelo que foram implementadas como tabelas na base de dados. Estas estão colocadas separadamente
pois não representam uma entidade no modelo ER, mas sim uma relação. Todas contêm as anotações \texttt{@Entity} e \texttt{@Table}, com o valor do campo \texttt{name}
igual à tabela correspondente da base de dados PostgreSQL.

Como todas estas são representadas através de identificadores de outras entidades, todas têm um campo com a anotação \texttt{@EmbeddedId}, sendo este campo do tipo
correspondente do \texttt{package} \texttt{embeddables}. A implementação destas classes encontra-se no anexo~\ref{a2}.

\section{Tables} \label{sec23}

Neste \textit{package} encontram-se as tabelas da base de dados que representam entidades no modelo ER. Estas fazem uso das classes presentes em relations para ser mais
fácil de aceder a certas informações, como, por exemplo, a lista de jogadores que compraram um jogo ou a lista de jogos que um jogador comprou. Estes estão presentes como
campos das classes \texttt{Jogo} e \texttt{Jogador}, respetivamente.

Existem ainda funções que acompanham estas implementações pois todos estes campos estão declarados como \texttt{private}, sendo apenas possível aceder fazendo uso de um
\textit{getter} e alterando o seu valor com um \textit{setter}.

A implementação destas classes encontra-se no anexo~\ref{a3}.

% Capitulo 3
%
% Capítulo 3
%
\chapter{Funcionalidade da Aplicação} \label{cap3}

Uma vez o modelo feito é agora possível proceder à implementação das funcionalidades desejadas do trabalho. Estas requerem primeiro acesso às funções e procedimentos
armazenados criados na primeira parte, os quais vêm em formato das alíneas do primeiro enunciado 2d até 2l.

Com o acesso a estas funcionalidades obtido, procede-se à implementação da alínea 2h como na fase 1 mas sem usar qualquer procedimento armazenado ou função pgSql,
ou seja, limitando as possibilidades para usar apenas interações entre \texttt{JPA} e PostgreSQL. Após esta implementação é realizada outra vez mas sem a limitação,
o qual permite facilmente comparar as duas implementações e concluir qual será melhor.

Esta aplicação também permite aumentar em 20\% o número de pontos associados a um crachá, o qual foi implementado com \textit{optimistic locking} e \textit{pessimistic
locking}. A versão que utiliza \textit{optimistic locking} vem acompanhada de testes que verificam que uma mensagem de erro é levantada quando existe uma alteração
concorrente que inviabilize a operação.

%
% Secção 3.1
%
\section{Acesso às Funcionalidades da Base de Dados} \label{sec31}

O primeiro requisito é disponibilizar as funcionalidades presentes na base de dados, començando com o exercício \texttt{2d} da primeira fase até ao exercício \texttt{2l}.
Estas são apresentadas na forma de funções da aplicação.

%
% Secção 3.1.1
%
\subsection{Funções}\label{sec311}

Nesta secção encontram-se as funções chamadas tal como na base de dados PostgreSQL.

\paragraph{Exercício 2d} Este exercício consiste em duas funções distintas, tal como na base de dados, pois estas eftuam operações diferentes. As duas funções são
\texttt{createPlayer} e \texttt{setPlayerState}. Nas listagens~\ref{lst:createPlayer} e \ref{lst:setPlayerState} encontram-se as funções \texttt{createPlayer} e
\texttt{setPlayerState}, respetivamente. Realça-se nestas implementações que a forma de chamar um função em PostgreSQL é como realizar uma \textit{query} para a 
função, tal como a \textit{string} criada na primeira linha de cada função.
\begin{minted}
[
    frame=lines,
    framesep=2mm,
    baselinestretch=1.2,
    fontsize=\footnotesize,
    linenos
]
{Java}
public String createPlayer(String username, String email, String regiao) {
    String query = "SELECT createPlayer(?1, ?2, ?3)";
    Query functionQuery = em.createNativeQuery(query)
            .setParameter(1, username)
            .setParameter(2, email)
            .setParameter(3, regiao);
    return (String) functionQuery.getSingleResult();
}
\end{minted}
\captionof{listing}{Código da função createPlayer\label{lst:createPlayer}}

\begin{minted}
[
    frame=lines,
    framesep=2mm,
    baselinestretch=1.2,
    fontsize=\footnotesize,
    linenos
]
{Java}
public String setPlayerState(int idJogador, String newState) {
    String query = "SELECT setPlayerState(?1, ?2)";
    Query functionQuery = em.createNativeQuery(query)
            .setParameter(1, idJogador)
            .setParameter(2, newState);
    return (String) functionQuery.getSingleResult();
}
\end{minted}
\captionof{listing}{Código da função setPlayerState\label{lst:setPlayerState}}

\paragraph{Exercício 2e} Este exercício apenas requer a implementação de uma função. Esta também tem apenas como objetivo chamar uma função da base de dados e não
requer nível de isolamente acima do que está por definição em PostgreSQL. Tal como nas funções em 2d, esta apenas realiza uma \textit{query} para a função na
base de dados. A implementação encontra-se na listagem~\ref{lst:totalPontosJogador}.
\begin{minted}
[
    frame=lines,
    framesep=2mm,
    baselinestretch=1.2,
    fontsize=\footnotesize,
    linenos
]
{Java}
public Long totalPontosJogador(int idJogador) {
    String query = "SELECT totalPontos from totalPontosJogador(?1)";
    Query functionQuery = em.createNativeQuery(query);
    functionQuery.setParameter(1, idJogador);
    return (Long) functionQuery.getSingleResult();
}
\end{minted}
\captionof{listing}{Código da função totalPontosJogador\label{lst:totalPontosJogador}}

\paragraph{Exercício 2f} Tal como as outras funções, esta é chamada através de uma \textit{query}. Este exercício pretende permitir saber quantos pontos um jogador
obteve no total das suas partidas em vários jogos.
\begin{minted}
[
    frame=lines,
    framesep=2mm,
    baselinestretch=1.2,
    fontsize=\footnotesize,
    linenos
]
{Java}
public Long totalJogosJogador(String email) {
    int idJogador = getPlayerByEmail(email).getId();
    String query = "SELECT totalJogos from totalJogosJogador(?1)";
    Query functionQuery = em.createNativeQuery(query);
    functionQuery.setParameter(1, idJogador);
    return (Long) functionQuery.getSingleResult();
}
\end{minted}
\captionof{listing}{Código da função totalJogosJogador\label{lst:totalJogosJogador}}

\paragraph{Exercício 2g} Esta função é mais complexa que as outras devido ao facto de retornar uma tabela e não apenas um valor ou o resultado de uma operação e não
há entidade para este resultado. Para resolver o problema é possível seguir várias opções: usar a função \texttt{getResultList} e colocar numa lista de arrays de objetos
da classe \texttt{Object}, fazer duas queries, uma para os jogadores e outra para as pontuações e depois agrupá-las num objeto da classe \texttt{Map}, ou criar uma
entidade apenas para os resultados desta função. A solução escolhida foi a primeira, tal como implementado na listagem~\ref{lst:PontosJogosPorJogador}
\begin{minted}
[
    frame=lines,
    framesep=2mm,
    baselinestretch=1.2,
    fontsize=\footnotesize,
    linenos
]
{Java}
public Map<Integer, BigDecimal> PontosJogosPorJogador(String gameName) {
    String idJogo = getGameByName(gameName).getId();
    String queryString = "SELECT jogadores, pontuacaoTotal from PontosJogosPorJogador(?1)";
    Query query = em.createNativeQuery(queryString);
    query.setParameter(1, idJogo);
    Map<Integer, BigDecimal> map = new java.util.HashMap<>(Map.of());
    List<Object[]> list  = query.getResultList();
    for (Object[] obj : list) {
        Integer idJogador = (Integer) obj[0];
        BigDecimal points = (BigDecimal) obj[1];
        map.put(idJogador,  points);
    }
    return map;
}
\end{minted}
\captionof{listing}{Código da função PontosJogosPorJogador\label{lst:PontosJogosPorJogador}}

%
% Secção 3.1.2
%
\subsection{Procedimentos armazenados}\label{sec312}

Esta secção cobre a chamada aos procedimentos armazenados da base de dados. Estes requerem um tratamento diferente pois podem ter níveis de transação específicos.

\paragraph{Exercício 2h} Este procedimento da base de dados permite associar um crachá a um jogador dados o identificador do jogador, o identificador do jogo e o
nome do crachá. Esta transação necessita de um nível de isolamento \texttt{TRANSACTION\_REPEATABLE\_READ}, tal como se identifica na linha 6 da listagem~\ref{lst:associarCracha}.
No final da transação é necessário fazer \texttt{transaction.commit()}, tal como na linha 12. Caso falhe, irá fazer \texttt{transaction.rollback()}, presente na linha 15.
\begin{minted}
[
    frame=lines,
    framesep=2mm,
    baselinestretch=1.2,
    fontsize=\footnotesize,
    linenos
]
{Java}
public void associarCracha(int idJogador, String gameName, String nomeCracha) {
    EntityTransaction transaction = startTransaction();
    Connection cn = em.unwrap(Connection.class);
    String idJogo = getGameByName(gameName).getNome();
    try {
        setIsolationLevel(cn, Connection.TRANSACTION_REPEATABLE_READ, transaction);
        try (CallableStatement storedProcedure = cn.prepareCall("call associarCracha(?,?, ?)")) {
            storedProcedure.setInt(1, idJogador);
            storedProcedure.setString(2, idJogo);
            storedProcedure.setString(3, nomeCracha);
            storedProcedure.executeUpdate();
            transaction.commit();
        }
    } catch(Exception e){
        if(transaction.isActive()) transaction.rollback();
    }
}
\end{minted}
\captionof{listing}{Código da função associarCracha\label{lst:associarCracha}}

\paragraph{Exercício 2i} Assim como no exercício 2h, este procedimento armazenado necessita de um nível de isolamento \texttt{TRANSACTION\_REPEATABLE\_READ}. Este procedimento
cria uma nova conversa e coloca o jogador que a iniciou dentro dela. Este procedimento coloca o identificador da conversa no seu terceiro argumento, o qual é criado na linha 4
da listagem~\ref{lst:iniciarConversa}. Este valor é depois retornado. Caso a transação falhe, retorna explicitamente \texttt{null}, pois não se garante que \texttt{idConversa}
permaneça com o valor \texttt{null} em caso de falha e consequente \texttt{rollback}.
\begin{minted}
[
    frame=lines,
    framesep=2mm,
    baselinestretch=1.2,
    fontsize=\footnotesize,
    linenos
]
{Java}
public Integer iniciarConversa(int idJogador, String nomeConversa) {
    EntityTransaction transaction = modelManager.startTransaction();
    Connection cn = em.unwrap(Connection.class);
    Integer idConversa = null;
    try {
        modelManager.setIsolationLevel(cn, Connection.TRANSACTION_REPEATABLE_READ, transaction);
        try (CallableStatement storedProcedure = cn.prepareCall("call iniciarConversa(?,?, ?)")) {
            storedProcedure.setInt(1, idJogador);
            storedProcedure.setString(2, nomeConversa);
            storedProcedure.registerOutParameter(3, Types.INTEGER);
            storedProcedure.executeUpdate();
            idConversa = storedProcedure.getInt(3);
            transaction.commit();
        }
    } catch(Exception e){
        if(transaction.isActive()) transaction.rollback();
        return null;
    }
    return idConversa;
}
\end{minted}
\captionof{listing}{Código da função iniciarConversa\label{lst:iniciarConversa}}

\paragraph{Exercício 2j} Este procedimento é dos mais simples, sendo que apenas necessita de um nível de isolamento \texttt{TRANSACTION\_READ\_COMMITTED}. Com o identificador
de um jogador e o identificador de uma conversa, colocará esse par na tabela \texttt{jogador\_conversa}, a qual indica que jogadores pertencem a cada conversa.
\begin{minted}
[
    frame=lines,
    framesep=2mm,
    baselinestretch=1.2,
    fontsize=\footnotesize,
    linenos
]
{Java}
public void juntarConversa(int idJogador, int idConversa) {
    EntityTransaction transaction = startTransaction();
    Connection cn = em.unwrap(Connection.class);
    try {
        try (CallableStatement storedProcedure = cn.prepareCall("call juntarConversa(?,?)")) {
            storedProcedure.setInt(1, idJogador);
            storedProcedure.setInt(2, idConversa);
            transaction.commit();
        }
    } catch(Exception e){
        if(transaction.isActive()) transaction.rollback();
    }
}
\end{minted}
\captionof{listing}{Código da função juntarConversa\label{lst:juntarConversa}}

\paragraph{Exercício 2k} Neste exercício o procedimento é, tal como em 2j, mais simples e requer apenas um nível de isolamento \texttt{TRANSACTION\_READ\_COMMITTED}. Usando
o identificador de um jogador, o qual envia a mensagem, o identificador da conversa para onde a mensagem é enviada e o conteúdo da mensagem em questão. Esta função apenas
cria uma transação e realiza \texttt{commit} em caso de sucesso e \texttt{rollback} em caso de falha.
\begin{minted}
[
    frame=lines,
    framesep=2mm,
    baselinestretch=1.2,
    fontsize=\footnotesize,
    linenos
]
{Java}
public void enviarMensagem(int idJogador, int idConversa, String content) {
    EntityTransaction transaction = modelManager.startTransaction();
    Connection cn = em.unwrap(Connection.class);
    try {
        modelManager.setIsolationLevel(cn, Connection.TRANSACTION_READ_COMMITTED, transaction);
        try (CallableStatement storedProcedure = cn.prepareCall("call enviarMensagem(?,?, ?)")) {
            cn.setTransactionIsolation(Connection.TRANSACTION_READ_COMMITTED);
            storedProcedure.setInt(1, idJogador);
            storedProcedure.setInt(2, idConversa);
            storedProcedure.setString(3, content);
            storedProcedure.executeUpdate();
            transaction.commit();
        }
    } catch(Exception e){
        if(transaction.isActive()) transaction.rollback();
    }
}
\end{minted}
\captionof{listing}{Código da função enviarMensagem\label{lst:enviarMensagem}}

%
% Secção 3.1.3
%
\subsection{Vistas} \label{sec313}

Esta última subsecção representa a implementação da vista presente na base de dados.

\paragraph{Exercício 2l} Esta vista permite obter a informação de todos os jogadores com exceção dos que têm o estado \texttt{'banido'}. Para obter os valores desta tabela optou-se
por criar uma entidade que representa estes resultados, pois uma vista funciona tal como uma tabela e, devido a esta característica, é efetuada uma \textit{query} normal. Após obter
os resultados esta função coloca-os no \textit{standard output}.
\begin{minted}
[
    frame=lines,
    framesep=2mm,
    baselinestretch=1.2,
    fontsize=\footnotesize,
    linenos
]
{Java}
public void jogadorTotalInfo(){
    String query = "Select * from jogadorTotalInfo";
    Query q = em.createNativeQuery(query, JogadorTotalInfo.class);
    List<JogadorTotalInfo> allInfo =  q.getResultList();
    for(JogadorTotalInfo jogador: allInfo){
        System.out.println(
                jogador.getEstado() + " " +
                jogador.getEmail() + " " +
                jogador.getUsername() + " " +
                jogador.getJogosParticipados() + " " +
                jogador.getPartidasParticipadas() + " " +
                jogador.getPontuacaoTotal() + " "
        );
    }
}
\end{minted}
\captionof{listing}{Código da função jogadorTotalInfo\label{lst:jogadorTotalInfo}}

%
% Secção 3.2
%
\section{Realização da Funcionalidade 2h} \label{sec32}

TODO(Descrever a funcionalidade)

%
% Secção 3.2.1
%
\subsection{Sem usar procedimentos e funções pgSql} \label{sec321}

TODO(Descrever a implementação sem pgSql)

%
% Secção 3.2.2
%
\subsection{Usando procedimentos e funções pgSql} \label{sec322}

TODO(Descrever a implementação com pgSql)

%
% Secção 3.3
%
\section{Funcionalidade Adicional} \label{sec33}

TODO(Aumentar a 20\% cena)

%
% Secção 3.3.1
%
\subsection{Implementação Com \textit{Optimistic Locking}} \label{sec331}

TODO(Descrever a implementação que usa optimistic locking)

%
% Secção 3.3.2
%
\subsection{Implementação Com \textit{Pessimistic Locking}} \label{sec332}

TODO(Descrever a implementação que usa pessimistic locking)

%
% Secção 3.3.3
%
\subsection{Testes da Funcionalidade}\label{sec333}

TODO(Descrever os testes e como se testou concorrência)

% Capitulo 4
%
% Capítulo 4
%
\chapter{Conclusão} \label{cap4}

Enquanto que uma base de dados por si só apresenta muito potencial, acompanhada de uma aplicação torna-se mais compreensível e fácil de usar. A segunda fase
do trabalho prático serviu para desenvolver esta aplicação e possibilitar o seu uso num ambiente de desenvolvimento Java.

Os objetivos foram todos cumpridos e permitem concluir que uma base de dados, ainda que apresente muito potencial, com o desenvolvimento de uma aplicação que
a acompanhe demonstra o seu verdadeiro potencial. Afirma-se também que há diferentes formas de realizar certas operações, sendo que a escolha depende apenas
de cada situação, não havendo uma solução que sirva para todos os casos.

%
% Secção 4.1
%
\section{Aspetos a melhorar} \label{sec31}

O modelo SQL desenvolvido na primeira fase do trabalho sofreu alterações devido à aplicação. Estas deveriam ter sido identificadas logo na fase 1. Um bom modelo
SQL deve procurar ser fácil de entender e de utilizar. A alteração em questão foi a adição da restrição \texttt{unique} à coluna \texttt{email} da tabela de jogadores,
desta forma é possível identificar um jogador através do seu \textit{email}.

% Referências
\bibliographystyle{unsrt}
\bibliography{referencias}

% Anexos (opcional)
\appendix

% Anexo A
%
% Anexo 1
%
\chapter{Código das Classes do \textit{Package} \texttt{model.embeddables}} \label{a1}
Neste anexo encontra-se a implementação das várias classes do \textit{package} \texttt{Embeddables}

\begin{minted}
[
    frame=lines,
    framesep=2mm,
    baselinestretch=1.2,
    fontsize=\footnotesize,
    linenos
]
{Java}
package model.embeddables;

import jakarta.persistence.Embeddable;
import java.io.Serializable;

// Composite key for table Compra
@Embeddable
public class CompraId implements Serializable {
    private int jogador;
    private String jogo;

    public CompraId(){}
    public void setJogoId(String jogo) {
        this.jogo = jogo;
    }

    public String getJogoId() {
        return jogo;
    }

    public void setJogadorId(int jogador) {
        this.jogador = jogador;
    }

    public int getJogadorId() {
        return jogador;
    }
}
\end{minted}
\captionof{listing}{Código da classe CompraId\label{lst:CompraId}}

\begin{minted}
[
    frame=lines,
    framesep=2mm,
    baselinestretch=1.2,
    fontsize=\footnotesize,
    linenos
]
{Java}
package model.embeddables;

import jakarta.persistence.Embeddable;

import java.io.Serializable;

// Composite jey for table cracha
@Embeddable
public class CrachaId implements Serializable {
    private String nome;
    private String jogo;

    public CrachaId(){}
    public String getNome() {
        return nome;
    }

    public void setNome(String nome) {
        this.nome = nome;
    }

    public String getJogo() {
        return jogo;
    }

    public void setJogo(String jogo) {
        this.jogo = jogo;
    }

}
\end{minted}
\captionof{listing}{Código da classe CrachaId\label{lst:CrachaId}}

\begin{minted}
[
    frame=lines,
    framesep=2mm,
    baselinestretch=1.2,
    fontsize=\footnotesize,
    linenos
]
{Java}
package model.embeddables;

import jakarta.persistence.Embeddable;

import java.io.Serializable;

@Embeddable
public class CrachasAdquiridosId implements Serializable {
    private int jogo;
    private String jogador;
    private String cracha;

    public CrachasAdquiridosId(){}
    public void setJogo(int jogo) {
        this.jogo = jogo;
    }

    public int getJogo() {
        return jogo;
    }

    public void setJogador(String jogador) {
        this.jogador = jogador;
    }

    public String getJogador() {
        return jogador;
    }

    public void setCracha(String cracha) {
        this.cracha = cracha;
    }

    public String getCracha() {
        return cracha;
    }
}
\end{minted}
\captionof{listing}{Código da classe CrachasAdquiridosId\label{lst:CrachasAdquiridosId}}

\begin{minted}
[
    frame=lines,
    framesep=2mm,
    baselinestretch=1.2,
    fontsize=\footnotesize,
    linenos
]
{Java}
package model.embeddables;

import jakarta.persistence.Embeddable;
import jakarta.persistence.GeneratedValue;
import jakarta.persistence.GenerationType;

import java.io.Serializable;

//Composite key for table mensagem
@Embeddable
public class MensagemId implements Serializable {
    @GeneratedValue(strategy = GenerationType.IDENTITY)
    private int id;
    private int conversa;

    public MensagemId(){ }
    public void setid(int id) {
        this.id = id;
    }

    public int getid() {
        return id;
    }

    public int getConversa() {
        return conversa;
    }
    public void setConversa(int conversa) {
        this.conversa = conversa;
    }

}
\end{minted}
\captionof{listing}{Código da classe MensagemId\label{lst:MensagemId}}

\begin{minted}
[
    frame=lines,
    framesep=2mm,
    baselinestretch=1.2,
    fontsize=\footnotesize,
    linenos
]
{Java}
package model.embeddables;

import jakarta.persistence.Embeddable;

import java.io.Serializable;

//Composite key for jogador_conversa rewlation participa
@Embeddable
public class ParticipaId implements Serializable {
    private int jogador;
    private int conversa;

    public ParticipaId(){}

    public void setJogadorId(int jogador) {
        this.jogador = jogador;
    }

    public int getJogadorId() {
        return jogador;
    }

    public void setConversaId(int conversa) {
        this.conversa = conversa;
    }

    public int getConversaId() {
        return conversa;
    }
}
\end{minted}
\captionof{listing}{Código da classe ParticipaId\label{lst:ParticipaId}}

\begin{minted}
[
    frame=lines,
    framesep=2mm,
    baselinestretch=1.2,
    fontsize=\footnotesize,
    linenos
]
{Java}
package model.embeddables;

import jakarta.persistence.Embeddable;
import jakarta.persistence.GeneratedValue;
import jakarta.persistence.GenerationType;

import java.io.Serializable;

// Composite Primary key for Partida
@Embeddable
public class PartidaId implements Serializable {
    @GeneratedValue(strategy = GenerationType.IDENTITY)
    private int partida;
    private String jogo;

    public PartidaId(){}
    public void setPartidaId(int partida) {
        this.partida = partida;
    }

    public int getPartidaId() {
        return partida;
    }

    public void setJogoId(String jogo) {
        this.jogo = jogo;
    }

    public String getJogoId() {
        return jogo;
    }
}
\end{minted}
\captionof{listing}{Código da classe PartidaId\label{lst:PartidaId}}

\begin{minted}
[
    frame=lines,
    framesep=2mm,
    baselinestretch=1.2,
    fontsize=\footnotesize,
    linenos
]
{Java}
package model.embeddables;

import jakarta.persistence.Embeddable;

import java.io.Serializable;

// Composite Primary key for Partida_Normal and Partida_MultiJogador
@Embeddable
public class PartidaNMId implements Serializable {
    int partida;
    String jogo;
    int jogador;

    public PartidaNMId(){}

    public void setPartida(int partidaId) {
        this.partida = partidaId;
    }

    public int getPartida() {
        return partida;
    }

    public String getJogo() {
        return jogo;
    }

    public int getJogador() {
        return jogador;
    }

    public void setJogador(int jogadorId) {
        this.jogador = jogadorId;
    }

    public void setJogo(String jogoId) {
        this.jogo = jogoId;
    }
}
\end{minted}
\captionof{listing}{Código da classe PartidaNMId\label{lst:PartidaNMId}}

% Anexo B
%
% Anexo 2
%
\chapter{Código das Classes do \textit{Package} \texttt{model.relations}} \label{a2}

Estas listagens contêm código correspondente às classes presentes no \textit{package} \texttt{relations}

\begin{minted}
[
    frame=lines,
    framesep=2mm,
    baselinestretch=1.2,
    fontsize=\footnotesize,
    linenos
]
{Java}
package model.relations;

import jakarta.persistence.*;
import model.tables.Jogador;
import model.tables.Jogo;
import model.embeddables.CompraId;

import java.io.Serial;
import java.io.Serializable;
import java.util.Date;

@Entity
@Table(name="compra")
public class Compra implements Serializable {

    @Serial
    private static final long serialVersionUID = 1L;

    public Compra() { }

    @EmbeddedId
    private CompraId id;
    private Date data;
    private Double preco;


    // N:N relation Compra between jogo and jogador
    @ManyToOne
    @MapsId("jogo")
    @JoinColumn(name = "jogo")
    private Jogo jogo;

    @ManyToOne
    @MapsId("jogador")
    @JoinColumn(name = "jogador")
    private Jogador jogador;

    //getter and setters
    public Jogador getJogador() { return jogador; }

    public void setJogador(Jogador jogador) { this.jogador = jogador; }

    public Jogo getJogo() { return jogo; }

    public void setJogo(Jogo jogo) { this.jogo = jogo; }

    public CompraId getId() { return this.id; }

    public Date getData() { return data; }

    public void setData(Date data) { this.data = data; }

    public void setPreco(Double preco) { this.preco = preco;}
    public Double getPreco() { return preco; }

}
\end{minted}
\captionof{listing}{Código da classe Compra\label{lst:Compra}}

\begin{minted}
[
    frame=lines,
    framesep=2mm,
    baselinestretch=1.2,
    fontsize=\footnotesize,
    linenos
]
{Java}
package model.relations;

import jakarta.persistence.*;
import model.tables.Cracha;
import model.tables.Jogador;
import model.embeddables.CrachasAdquiridosId;

import java.io.Serializable;

@Entity
@Table(name="crachá_jogador")
public class CrachasAdquiridos implements Serializable {

    @EmbeddedId
    private CrachasAdquiridosId id;


    // N:N realtion CrachasAdquiridos
    @ManyToOne
    @MapsId("jogador")
    @JoinColumn(name="jogador",  referencedColumnName = "jogador")
    private Jogador jogador;


    @ManyToOne
    @JoinColumns({
            @JoinColumn(name = "cracha_nome", referencedColumnName = "nome"),
            @JoinColumn(name = "cracha_jogador", referencedColumnName = "jogador")
    })
    private Cracha cracha;


    public CrachasAdquiridosId getId() {
        return id;
    }

    public void setId(CrachasAdquiridosId id) {
        this.id = id;
    }

    public Cracha getCracha() {
        return cracha;
    }

    public void setCracha(Cracha cracha) {
        this.cracha = cracha;
    }

    public Jogador getJogador() {
        return jogador;
    }

    public void setJogador(Jogador jogador) {
        this.jogador = jogador;
    }
}
\end{minted}
\captionof{listing}{Código da classe CrachasAdquiridos\label{lst:CrachasAdquiridos}}

\begin{minted}
[
    frame=lines,
    framesep=2mm,
    baselinestretch=1.2,
    fontsize=\footnotesize,
    linenos
]
{Java}
package model.relations;

import jakarta.persistence.*;
import model.tables.Conversa;
import model.tables.Jogador;
import model.embeddables.ParticipaId;

import java.io.Serial;
import java.io.Serializable;

@Entity
@Table(name = "jogador_conversa")
public class Participa implements Serializable {

    @Serial
    private static final long serialVersionUID = 1L;

    public Participa(){ }
    @EmbeddedId
    private ParticipaId id;

    @ManyToOne
    @MapsId("jogador")
    @JoinColumn(name="jogador")
    private Jogador jogador;

    @ManyToOne
    @MapsId("conversa")
    @JoinColumn(name="conversa")
    private Conversa conversa;


    //getters and setters
    public ParticipaId getId() {return id;}

    public void setId(ParticipaId id) { this.id = id; }

    public Jogador getJogador() { return jogador; }

    public void setJogador(Jogador jogador) { this.jogador = jogador; }

    public Conversa getConversa() { return conversa; }

    public void setConversa(Conversa conversa) { this.conversa = conversa; }
}
\end{minted}
\captionof{listing}{Código da classe Participa\label{lst:Participa}}

% Anexo C
%
% Anexo 2
%
\chapter{Código das Classes do \textit{Package} \texttt{model.tables}} \label{a3}
\begin{minted}
[
    frame=lines,
    framesep=2mm,
    baselinestretch=1.2,
    fontsize=\footnotesize,
    linenos
]
{Java}
package model.tables;

import jakarta.persistence.*;
import model.relations.Compra;

import java.io.Serial;
import java.io.Serializable;
import java.util.ArrayList;
import java.util.List;

@Entity
@Table(name="jogo")
public class Jogo implements Serializable {

    @Serial
    private static final long serialVersionUID = 1L;

    @Id
    @GeneratedValue(strategy = GenerationType.UUID)
    private String id;
    private String nome;
    private String url;

    @OneToMany(mappedBy = "jogo")
    private List<Compra> compras = new ArrayList<>();

    public void addCompra(Compra compra){ this.compras.add(compra); }

    public List<Compra> getCompras() { return compras; }


    // 1:N pertence
    @OneToMany(mappedBy="jogo",cascade=CascadeType.PERSIST, orphanRemoval=true)
    private List<Cracha> crachas = new ArrayList<>();

    public List<Cracha> getCrachas() {
        return crachas;
    }

    public void addCracha(Cracha cracha) {
        this.crachas.add(cracha);
    }
    @OneToMany(mappedBy = "jogo", cascade = CascadeType.PERSIST, orphanRemoval=true)
    private List<Partida> partidas = new ArrayList<>();

    public List<Partida> getPartidas() {
        return partidas;
    }

    public void addPartida(Partida partida){
        partidas.add(partida);
    }

    public String getId(){ return this.id; }
    public String getNome(){ return this.nome; }
    public void setNome(String nome) { this.nome = nome; }

    public String getUrl(){ return this.url; }
    public void setUrl(String url) { this.url = url; }

}
\end{minted}
\captionof{listing}{Código da classe Jogo\label{lst:Jogo}}

\begin{minted}
[
    frame=lines,
    framesep=2mm,
    baselinestretch=1.2,
    fontsize=\footnotesize,
    linenos
]
{Java}
package model.tables;

import jakarta.persistence.*;
import model.relations.Compra;
import model.relations.CrachasAdquiridos;
import model.relations.Participa;

import java.io.Serial;
import java.io.Serializable;
import java.util.ArrayList;
import java.util.List;

@Entity
@Table(name="jogador")
public class Jogador implements Serializable {
    @Serial
    private static final long serialVersionUID = 1L;

    public Jogador(){ }


    @Id
    @GeneratedValue(strategy = GenerationType.IDENTITY)
    private int id;
    private String email;
    private String username;
    private String estado;
    private String regiao;
    @OneToMany(mappedBy = "jogador")
    private List<Compra> compras = new ArrayList<>();
    public void addCompra(Compra compra){ this.compras.add(compra); }
    public List<Compra> getCompras() { return compras; }


    //N:N Crachas

    @OneToMany(mappedBy = "jogador")
    private List<CrachasAdquiridos> crachasAdquiridosList = new ArrayList<>();

    public List<CrachasAdquiridos> getCrachasList() {
        return crachasAdquiridosList;
    }

    public void adquirirCracha(CrachasAdquiridos crachasAdquiridos) {
        this.crachasAdquiridosList.add(crachasAdquiridos);
    }


    //N:N Participa Relation
    @OneToMany(mappedBy = "jogador")
    private List<Participa> participaList = new ArrayList<>();
    public List<Participa> getParticipaList() { return this.participaList; }
    public void addParticipacao(Participa participa){this.participaList.add(participa);}

    @OneToMany(mappedBy = "jogador", cascade = CascadeType.PERSIST)
    private List<Partida_Normal> partidasNormais = new ArrayList<>();

    public List<Partida_Normal> getPartidasNormais() {
        return partidasNormais;
    }

    public void addPartida_Normal(Partida_Normal partida_normal){
        this.partidasNormais.add(partida_normal);
    }

    @OneToMany(mappedBy = "jogador", cascade = CascadeType.PERSIST)
    private List<Partida_MultiJogador> partidasMultiJogador = new ArrayList<>();

    public void addPartidaMultiJogador(Partida_MultiJogador partida){
        this.partidasMultiJogador.add(partida);
    }
    public List<Partida_MultiJogador> getPartidasMultiJogador() {
        return partidasMultiJogador;
    }



    //getter and setters
    public int getId(){
        return this.id;
    }

    public String getEmail(){
        return this.email;
    }

    public void setEmail(String email){
        this.email = email;
    }

    public String getUsername(){
        return this.username;
    }

    public void setUsername(String username){
        this.username = username;
    }

    public String getEstado(){
        return this.estado;
    }

    public void setEstado(String estado){
        this.estado = estado;
    }

    public String getRegiao(){
        return this.regiao;
    }

    public void setRegiao(String regiao){
        this.regiao = regiao;
    }

}
\end{minted}
\captionof{listing}{Código da classe Jogador\label{lst:Jogador}}

\begin{minted}
[
    frame=lines,
    framesep=2mm,
    baselinestretch=1.2,
    fontsize=\footnotesize,
    linenos
]
{Java}
package model.tables;

import jakarta.persistence.*;
import model.embeddables.PartidaId;

import java.io.Serial;
import java.io.Serializable;
import java.util.Date;

@Entity
@Table(name="partida")
public class Partida implements Serializable {

    @Serial
    private static final long serialVersionUID = 1L;

    public Partida() { }

    @EmbeddedId
    private PartidaId id;

    String estado;
    String nome_regiao;

    Date dtinicio;

    Date dtfim;

    // 1:N
    @ManyToOne
    @MapsId
    @JoinColumn(name = "id")
    private Jogo jogo;

    public void setJogo(Jogo jogo) {
        this.jogo = jogo;
    }

    public Jogo getJogo() {
        return jogo;
    }

    public PartidaId getId() {
        return id;
    }

    @OneToOne(mappedBy = "partida")
    private Partida_MultiJogador partida_multiJogador;

    public Partida_MultiJogador getPartida_multiJogador() {
        return partida_multiJogador;
    }

    public void setPartida_multiJogador(Partida_MultiJogador partida_multiJogador) {
        this.partida_multiJogador = partida_multiJogador;
    }

    @OneToOne(mappedBy = "partida")
    private Partida_Normal partida_normal;

    public Partida_Normal getPartida_normal() {
        return partida_normal;
    }

    public void setPartida_normal(Partida_Normal partida_normal) {
        this.partida_normal = partida_normal;
    }

    public void setId(PartidaId id) {
        this.id = id;
    }

    public String getEstado() {
        return estado;
    }

    public void setEstado(String estado) {
        this.estado = estado;
    }

    public String getNome_regiao() {
        return nome_regiao;
    }

    public void setNome_regiao(String nome_regiao) {
        this.nome_regiao = nome_regiao;
    }

    public Date getDtinicio() {
        return dtinicio;
    }

    public void setDtinicio(Date dtinicio) {
        this.dtinicio = dtinicio;
    }

    public void setDtfim(Date dtfim) {
        this.dtfim = dtfim;
    }

    public Date getDtfim() {
        return dtfim;
    }
}
\end{minted}
\captionof{listing}{Código da classe Partida\label{lst:Partida}}

\begin{minted}
[
    frame=lines,
    framesep=2mm,
    baselinestretch=1.2,
    fontsize=\footnotesize,
    linenos
]
{Java}
package model.tables;

import jakarta.persistence.*;
import model.embeddables.PartidaMId;
import model.embeddables.PartidaNId;

import java.io.Serial;
import java.io.Serializable;

@Entity
@Table(name="partida_multijogador")
public class Partida_MultiJogador implements Serializable {
    @Serial
    private static final long serialVersionUID = 1L;

    public Partida_MultiJogador() { }

    @EmbeddedId
    private PartidaMId partida_multiId;

    int pontuacao;

    @OneToOne
    @JoinColumns({
            @JoinColumn(name = "partida", referencedColumnName = "partida", insertable = false, updatable = false),
            @JoinColumn(name = "jogo", referencedColumnName = "jogo", insertable = false, updatable = false)
    })
    private Partida partida;


    @ManyToOne
    @MapsId
    @JoinColumn(name = "jogador", referencedColumnName = "jogador")
    private Jogador jogador;

    public Jogador getJogador() {
        return jogador;
    }

    public void setJogador(Jogador jogador) {
        this.jogador = jogador;
    }

    public Partida getPartida() {
        return partida;
    }

    public void setPontuacao(int pontuacao) {
        this.pontuacao = pontuacao;
    }

    public void setPartida_multiId(PartidaMId partida) {
        this.partida_multiId = partida;
    }

    public PartidaMId getPartida_multiId() {
        return partida_multiId;
    }

    public int getPontuacao() {
        return pontuacao;
    }

}
\end{minted}
\captionof{listing}{Código da classe Partida\_MultiJogador\label{lst:Partida_MultiJogador}}

\begin{minted}
[
    frame=lines,
    framesep=2mm,
    baselinestretch=1.2,
    fontsize=\footnotesize,
    linenos
]
{Java}
package model.tables;

import jakarta.persistence.*;
import model.embeddables.PartidaNId;

import java.io.Serial;
import java.io.Serializable;

@Entity
@Table(name="partida_normal")
public class Partida_Normal extends Partida implements Serializable {

    @Serial
    private static final long serialVersionUID = 1L;

    public Partida_Normal() { }

    @EmbeddedId
    private PartidaNId partida_normalId;

    private int pontuacao;

    @OneToOne
    @JoinColumns({
            @JoinColumn(name = "partida", referencedColumnName = "partida", insertable = false, updatable = false),
            @JoinColumn(name = "jogo", referencedColumnName = "jogo", insertable = false, updatable = false)
    })
    private Partida partida;

    public Partida getPartida() {
        return partida;
    }

    @ManyToOne
    @MapsId
    @JoinColumn(name = "jogador", referencedColumnName = "jogador")
    private Jogador jogador;

    public Jogador getJogador() {
        return jogador;
    }

    public void setJogador(Jogador jogador) {
        this.jogador = jogador;
    }

    public PartidaNId getPartida_normalId() {
        return partida_normalId;
    }

    public void setPartida_normalId(PartidaNId partida) {
        this.partida_normalId = partida;
    }

    public int getPontuacao() {
        return pontuacao;
    }

    public void setPontuacao(int pontuacao) {
        this.pontuacao = pontuacao;
    }


}
\end{minted}
\captionof{listing}{Código da classe Partida\_Normal\label{lst:Partida_Normal}}

\begin{minted}
[
    frame=lines,
    framesep=2mm,
    baselinestretch=1.2,
    fontsize=\footnotesize,
    linenos
]
{Java}
package model.tables;


import jakarta.persistence.*;
import model.embeddables.CrachaId;
import model.relations.CrachasAdquiridos;

import java.io.Serial;
import java.io.Serializable;
import java.util.ArrayList;
import java.util.List;

@Table(name="cracha")
@Entity
public class Cracha implements Serializable {

    @Serial
    private static final long serialVersionUID = 1L;

    public static long getSerialVersionUID() {
        return serialVersionUID;
    }

    @EmbeddedId
    private CrachaId id;

    int limite;

    String url;

    @ManyToOne
    @MapsId
    @JoinColumn(name="jogo")
    private Jogo jogo;

    public Jogo getJogo() {
        return jogo;
    }

    public void setJogo(Jogo jogo) {
        this.jogo = jogo;
    }

    @OneToMany(mappedBy = "cracha")
    private List<CrachasAdquiridos> crachasAdquiridosList = new ArrayList<>();

    public List<CrachasAdquiridos> getCrachasList() {
        return crachasAdquiridosList;
    }

    public void addCrachas(CrachasAdquiridos crachasAdquiridos) {
        this.crachasAdquiridosList.add(crachasAdquiridos);
    }


    public CrachaId getId() {
        return this.id;
    }

    public void setId(CrachaId id) {
        this.id = id;
    }

    public int getLimite() {
        return limite;
    }

    public void setLimite(int limit) {
        this.limite = limit;
    }

    public String getUrl() {
        return url;
    }

    public void setUrl(String url) {
        this.url = url;
    }

}
\end{minted}
\captionof{listing}{Código da classe Cracha\label{lst:Cracha}}

\begin{minted}
[
    frame=lines,
    framesep=2mm,
    baselinestretch=1.2,
    fontsize=\footnotesize,
    linenos
]
{Java}
package model.tables;

import jakarta.persistence.*;
import model.relations.Participa;

import java.io.Serial;
import java.io.Serializable;
import java.util.ArrayList;
import java.util.List;

@Entity
@Table(name="conversa")
public class Conversa implements Serializable {

    @Serial
    private static final long serialVersionUID = 1L;

    public Conversa() { }

    @Id
    @GeneratedValue(strategy = GenerationType.IDENTITY)
    private int id;
    private String nome;

    // N:N relation participa
    @OneToMany(mappedBy = "conversa")
    private List<Participa> participantes = new ArrayList<>();

    public List<Participa> getParticipantes(){ return this.participantes; }

    public void addJogador(Participa participante){
        this.participantes.add(participante);
    }

    // 1:N relation contem
    @OneToMany(mappedBy = "conversa", cascade = CascadeType.PERSIST, orphanRemoval = true)
    private List<Mensagem> mensagens = new ArrayList<>();

    public List<Mensagem> getMensagens() {
        return mensagens;
    }

    public void addMensagem(Mensagem mensagem) {
        this.mensagens.add(mensagem);
    }

    //getter and setters
    public int getId() { return id; }
    public void setNome(String nome){ this.nome = nome; }
    public String getNome(){ return this.nome; }
}
\end{minted}
\captionof{listing}{Código da classe Conversa\label{lst:Conversa}}

\begin{minted}
[
    frame=lines,
    framesep=2mm,
    baselinestretch=1.2,
    fontsize=\footnotesize,
    linenos
]
{Java}
package model.tables;

import jakarta.persistence.*;
import model.embeddables.MensagemId;

import java.io.Serial;
import java.io.Serializable;
import java.sql.Date;

@Entity
@Table(name = "mensagem")
public class Mensagem implements Serializable {

    @Serial
    private static final long serialVersionUID = 1L;
    public Mensagem() { }

    @EmbeddedId
    private MensagemId mensagemId;

    String conteudo;

    Date data;


    @ManyToOne
    @MapsId
    @JoinColumn(name="conversa")
    private Conversa conversa;

    public MensagemId getMensagemId() {
        return mensagemId;
    }

    public void setMensagemId(MensagemId mensagemId) {
        this.mensagemId = mensagemId;
    }

    public Date getData() {
        return data;
    }

    public void setData(Date data) {
        this.data = data;
    }

    public String getConteudo() {
        return conteudo;
    }

    public void setConteudo(String conteudo) {
        this.conteudo = conteudo;
    }
}
\end{minted}
\captionof{listing}{Código da classe Mensagem\label{lst:Mensagem}}

\end{document}
