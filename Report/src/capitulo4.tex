%
% Capítulo 4
%
\chapter{Conclusão} \label{cap4}

Enquanto que uma base de dados por si só apresenta muito potencial, acompanhada de uma aplicação torna-se mais compreensível e fácil de usar. A segunda fase
do trabalho prático serviu para desenvolver esta aplicação e possibilitar o seu uso num ambiente de desenvolvimento Java.

Os objetivos foram todos cumpridos e permitem concluir que uma base de dados, ainda que apresente muito potencial, com o desenvolvimento de uma aplicação que
a acompanhe demonstra o seu verdadeiro potencial. Afirma-se também que há diferentes formas de realizar certas operações, sendo que a escolha depende apenas
de cada situação, não havendo uma solução que sirva para todos os casos.

%
% Secção 4.1
%
\section{Aspetos a melhorar} \label{sec31}

O modelo SQL desenvolvido na primeira fase do trabalho sofreu alterações devido à aplicação. Estas deveriam ter sido identificadas logo na fase 1. Um bom modelo
SQL deve procurar ser fácil de entender e de utilizar. A alteração em questão foi a adição da restrição \texttt{unique} à coluna \texttt{email} da tabela de jogadores,
desta forma é possível identificar um jogador através do seu \textit{email}.